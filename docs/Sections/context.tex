\section{Comprensión del Negocio}

Esta sección establece el marco contextual del proyecto, definiendo la
problemática actual en el ámbito del Internet de las Cosas (IoT), los objetivos
planteados y los criterios mediante los cuales se evaluará el éxito de la
solución propuesta.

\subsection{Contexto general}

El Internet de las Cosas (IoT) ha experimentado un crecimiento exponencial en
la última década, integrándose en sectores críticos como la industria, la salud
y las ciudades inteligentes. Sin embargo, esta proliferación de dispositivos
conectados ha ampliado significativamente la superficie de ataque disponible
para los ciberdelincuentes.

A diferencia de los sistemas informáticos tradicionales, los dispositivos IoT
suelen caracterizarse por tener recursos limitados de procesamiento y energía,
lo que dificulta la implementación de mecanismos de seguridad robustos
convencionales. Las vulnerabilidades comunes incluyen el uso de credenciales
predeterminadas, falta de cifrado en las comunicaciones y firmware
desactualizado.

En este escenario, los sistemas de seguridad perimetral tradicionales (como
firewalls básicos) resultan insuficientes. Surge entonces la necesidad crítica
de implementar Sistemas de Detección de Intrusos (IDS) basados en el análisis
de tráfico de red, capaces de monitorear el comportamiento de los dispositivos
y alertar sobre actividades sospechosas en tiempo real.

\subsection{Problema a resolver}

El problema central que aborda este proyecto es la dificultad para distinguir,
de manera automatizada y precisa, entre el comportamiento legítimo de un
dispositivo IoT y las anomalías generadas por ciberataques.

Específicamente, el desafío consiste en analizar flujos de tráfico de red
complejos para realizar una clasificación multicategoría. No basta con una
detección binaria (normal vs.~ataque); es necesario identificar la naturaleza
específica de la intrusión (por ejemplo, ataques de denegación de servicio,
escaneos de puertos o inyecciones de fuerza bruta) utilizando los datos
proporcionados por el dataset RT-IoT2022.

\subsection{Objetivo del proyecto}

El objetivo principal de esta investigación es desarrollar, entrenar y validar
un algoritmo de clasificación utilizando métodos de aprendizaje automático
(Machine Learning). Este modelo debe ser capaz de procesar las características
del tráfico de red contenidas en el dataset RT-IoT2022 para distinguir
eficazmente entre tráfico normal y tráfico malicioso.

Se busca obtener un modelo que no solo sea preciso, sino también eficiente,
considerando las restricciones de latencia y recursos típicas de los entornos
IoT.

\subsection{Métricas de éxito}

Para evaluar el desempeño y la viabilidad de los modelos propuestos, se
utilizarán las siguientes métricas estándar en la industria de la
ciberseguridad y la ciencia de datos:

\begin{itemize}
    \item \textbf{Accuracy (Exactitud):} Para medir el porcentaje global de predicciones correctas.
    \item \textbf{Precision (Precisión):} Para evaluar la confiabilidad de las alertas de ataque generadas (minimización de falsos positivos).
    \item \textbf{Recall (Sensibilidad):} Crucial en seguridad, para medir la capacidad del modelo de detectar la mayor cantidad posible de ataques reales (minimización de falsos negativos).
    \item \textbf{F1-Score:} Como media armónica entre Precision y Recall, proporcionando una métrica balanceada, especialmente útil dado el posible desbalance de clases en los datos de ataques

    \item \textbf{Matriz de Confusión:} Para visualizar detalladamente el desempeño del modelo en cada clase específica de ataque.
\end{itemize}